% !TeX root = ../main.tex

{
    \metroset{sectionpage=none}
    \section{Zuvor bei HRaCs...}
}

\begin{frame}{Zuvor bei HRaCS...}
    \begin{itemize}
        % Chapter 1
        \item Logik \& Psychologie
        \item normative, deskriptive, präskriptive Regeln
        \pause
        
        % Chapter 2
        \item Konzepte der modernen Logik, Klassische Aussagenlogik, Closed-World-Reasoning etc.
        \item großer Spielraum für Interpretationen von Konditionalen
        \pause

        % Chapter 3
        \item \textsc{Wason}'s Selection Task, Two-Rule-Task, Tutoren-Dialoge
    \end{itemize}
\end{frame}


\begin{frame}{Wason's Selection Task {\scriptsize \cite[S.~44-46]{stenningHumanReasoningCognitive2008}}}
    \textbf{Regel:} \emph{If there is a vowel on one side, then there is an even number on the other side.}

    \[
        \framebox[5em][c]{A\strut}\quad
        \framebox[5em][c]{K\strut}\quad
        \framebox[5em][c]{4\strut}\quad
        \framebox[5em][c]{7\strut}
    \]

    Welche Karten \emph{müssen} umgedreht werden, um zu entscheiden, ob die Regel wahr ist?

    $\Rightarrow (p \to q) \leadsto (\text{Vokal} \to 2\mathbb{N})$

    Kompetenzantwort: $p$ und $\lnot q$, d.h. A und 7
\end{frame}

\begin{frame}{Wason's Selection Task {\scriptsize \cite[S.~44-46]{stenningHumanReasoningCognitive2008}}}
    $$(p \to q) \leadsto (\text{Vokal} \to 2\mathbb{N})$$

    \begin{center}
        \begin{tabular}{c | c | c | c | c}
            $p$ &$p,~q$ &\alert{$p, ~\lnot q$} &$p, ~q, ~\lnot q$ &anderes \\
            \hline
            A &A, 4 &\alert{A, 7} &A, 4, 7 & \\
            \hline \hline
            35\% &45\% &\alert{5\%} &7\% &8\%
        \end{tabular}
    \end{center}
\end{frame}


\begin{frame}{Two-Rule Task {\scriptsize \cite[S.~61-63]{stenningHumanReasoningCognitive2008}}}
    \begin{enumerate}
        \item \emph{If there is a $U$ on one side, then there is an 8 on the other side.}
        \item \emph{If there is a $I$ on one side, then there is an 8 on the other side.}
    \end{enumerate}

    \[
        \framebox[5em][c]{U\strut}\quad
        \framebox[5em][c]{I\strut}\quad
        \framebox[5em][c]{8\strut}\quad
        \framebox[5em][c]{3\strut}
    \]

    Genau eine Regel ist wahr. Welche Karten \emph{müssen} umgedreht werden, um zu entscheiden, welche Regel wahr ist?

    $\Rightarrow (p \to q) \leadsto (\{U, I\} \to 8)$

    Kompetenzantwort: $\lnot q$, d.h. $3$\\
    {\footnotesize (hinreichend, identifiziert falsche Regel)}
\end{frame}


\begin{frame}{Zuvor bei HRaCS...}
    \begin{itemize}
        {
            \color{gray}
            \setbeamertemplate{itemize item}{\color{gray}{\textbullet}}

            % Chapter 1
            \item Logik \& Psychologie
            \item normative, deskriptive, präskriptive Regeln
            
            % Chapter 2
            \item Konzepte der modernen Logik, Klassische Aussagenlogik, Closed-World-Reasoning etc.
            \item großer Spielraum für Interpretationen von Konditionalen
        }

        % Chapter 3
        \item \textsc{Wason}'s Selection Task, Two-Rule-Task, Tutoren-Dialoge
        \item \textsc{Bayes}'sches Information-Gain Model
        \pause

        \item[$\Rightarrow$] \alert{Interpretationsschwierigkeiten} von Proband:innen
        \begin{itemize}
            \item v.a. bei (abstrakten) deskriptiven Konditionalen
            \item kaum/nicht bei deontischen Regeln mit vertrauten Thematiken
        \end{itemize}
    \end{itemize}
\end{frame}
