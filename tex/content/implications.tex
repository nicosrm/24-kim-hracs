% !TeX root = ../main.tex

\section{Implikationen}

\begin{frame}{Implikationen {\scriptsize \cite[S.~112-113]{stenningHumanReasoningCognitive2008}}}
    \begin{itemize}
        \item Auswirkugn auf evolutionäre Theorien  in Kapitel 6
        \item nicht-evolutionäre Theorien am stärksten von Ergebnissen betroffen,
            basierend auf Auswirkungen für Beziehung zwischen Logik und Psychologie
        \item Fokus auf Relevanz-Theorie, Theorie der mentalen Modelle und Rationale-Analyse-Theorie
    \end{itemize}
\end{frame}


\begin{frame}{Relevanz-Theorie {\scriptsize \cite[S.~113-114]{stenningHumanReasoningCognitive2008}}}
    \begin{itemize}
        \item RT: menschliches Schließen / Kommunikation als allgemeine, kontext-sensitive Fähigkeit
        \item Erklärung von Effekten durch allgemeine Faktoren
        \begin{itemize}
            \item Relevanz der Aufgabe
            \item Kosten für Ermittlung der Relevanz
            \item Unterschiede zwischen \emph{Deontics} und \emph{Descriptives} fehlend
        \end{itemize}
        \item RT: kein großartiges Schließen bei falscher Antwort \\
            $\lightning$ Autoren: \enquote{a great deal goes on, however speedily}
    \end{itemize}
\end{frame}


\begin{frame}{Theorie der mentalen Modelle {\scriptsize \cite[S.~113-114]{stenningHumanReasoningCognitive2008}}}
    \begin{itemize}
        \item aufbauend auf komplexer Theorie der
        \begin{itemize}
            \item Bedeutung von Konditionalen und
            \item kontext-basierten Anpassung der Semantik {\footnotesize \emph{Pragmatic Modulation}}
        \end{itemize}
        
        \item Frage: sind Interpretationen wahrheitsfunktional? {\scriptsize (vgl. Kapitel 2)}
        
        \item nicht wahrheitsfunktional $\overset?\to$ Interpretation $\lightning$ Anweisungen \\
            {\footnotesize (Konflikt nie von Theorie anerkannt)}
        \pause

        \item[$\Rightarrow$] Warum wird klassisches, logisches Kompetenzmodell für \alert{Bewertung der Leistung}, 
            aber nicht zur Erklärung des \alert{Verständnisses von Konditionalen} verwendet? {\footnotesize (Kapitel 5, 10)}
    \end{itemize}
\end{frame}


\begin{frame}{Information-Gain-Theorie {\scriptsize \cite[S.~114-115]{stenningHumanReasoningCognitive2008}}}
    \begin{itemize}
        \item ausgreifteste Theorie, stellt als einziges das klassische Kompetenzmodell infrage
        \item unterschiedliche Betrachtung von deskriptiver und deontischer Aufgabe notwendig
        \begin{itemize}
            \item Leistung der Proband:innen gleich korrekt bzgl. jeweiligem Kompetenzmodell
            \item[$\lightning$] Autoren: deskriptiv hoch problematisch, deontisch einfach
            \begin{itemize}
                \item gestützt auf Tutoren-Dialoge (Kapitel 3) und Experimenten
                \item Vorhersage der Leistung basierend auf angenommene Interpretation
            \end{itemize}
        \end{itemize}
        \item IG: abweisen jeglicher Rolle der Logik \\
            $\lightning$ Autoren: durch Interpretation der natürlichen Sprache enthalten
    \end{itemize}
\end{frame}


\begin{frame}{Allgemeine Implikationen {\scriptsize \cite[S.~115]{stenningHumanReasoningCognitive2008}}}
    \begin{itemize}
        \item Lorem Ipsum
    \end{itemize}
\end{frame}
