% !TeX root = ../main.tex

\section{Experimente}

\begin{frame}{Standard-Wason \& Two-Rule Task {\scriptsize \cite[S.~95-96]{stenningHumanReasoningCognitive2008}}}
    \begin{itemize}
        \item klassische \enquote{abstrakte} Aufgabe (\textsc{Wason}) \\
            $\Rightarrow$ Grundlage für Ergebnisse
        \pause

        \item Two-Rule Task \\
            $\Rightarrow$ Robustheit natürlicher-sprachlicher Konditionale \\
            $\Rightarrow$ Begriff \enquote{Wahrheit}, Ausnahmen? \\
            \pause
            $\Rightarrow$ Auswirkung \textsc{Bayes}'sche Information-Gain-Theorie
    \end{itemize}
\end{frame}


\begin{frame}{Kontigenz {\scriptsize \cite[S.~96-97]{stenningHumanReasoningCognitive2008}}}
    \emph{
        {\small Also below there appears a rule.
        Your task is to decide which of these four cards you must turn (if any) in order to decide if the rule is true.}
        \alert{Assume that you have to \textbf{decide} whether to turn each card \textbf{before you get any information} from any of the \textbf{turns} you choose to make.}
        {\small Don't turn unnecessary cards. Tick the cards you want to turn.}
    }

    \begin{itemize}
        \item Untersuchung von durch deskriptive Semantik eingeführte Kontingenzen {\footnotesize (Zusammenhänge, Contingency)}
        \item Entscheidung über \emph{alle} Karten treffen vor weiteren Informationen
        \pause
        \item These: häufigere Wahl von $\lnot q$
    \end{itemize}
\end{frame}


\begin{frame}{Truthfulness {\scriptsize \cite[S.~97-98]{stenningHumanReasoningCognitive2008}}}
    \emph{ \small
        Also below there appears a rule {\normalsize \alert{put forward by an \textbf{unreliable source}}}.
        Your task is to decide which cards (if any) you must turn in order to decide {\normalsize \alert{if the unreliable source is \textbf{lying}}}.
        Don't turn unnecessary cards. Tick the cards you want to turn.
    }
    
    Quelle der Regel $\ne$ Experimentator:in

    \pause
    \begin{itemize}
        \item[$\Rightarrow$] Untersuchung des Einflusses von
        \begin{itemize}
            \item autoritärer Position der Experimentator:innen
            \item Gleichgewicht zwischen kooperativer und gegnerischer Haltung
        \end{itemize}

        \item[$\Rightarrow$] \emph{Truthfulness} der \emph{Quelle} statt der Regel
    \end{itemize}
\end{frame}


\begin{frame}{Konjunktion {\scriptsize \cite[S.~99]{stenningHumanReasoningCognitive2008}}}
    \begin{itemize}
        \item Untersuchung von Interpretationen außerhalb von Konditionalen
            $\Rightarrow$ konjunktive Regeln ($\land$)
        
        \item Untersuchung der logischen Veränderung (keine Vereinfachung)
        
        \item[$\Rightarrow$] \emph{There is a vowel on one side, \alert{and} there is an even number on the other side.} ($p \land q$)
        
        \[
            \framebox[5em][c]{A\strut}\quad
            \framebox[5em][c]{K\strut}\quad
            \framebox[5em][c]{4\strut}\quad
            \framebox[5em][c]{7\strut}
        \]
    \end{itemize}

    \pause
    \begin{itemize}
        \item klassische Kompetenzantwort: \emph{keine} Karte umdrehen \\
            {\footnotesize (bereits widerlegt durch $\lnot p$ und $\lnot q$)}
        
        \item Vermutung: seltenes Vorkommen der Kompetenzantwort
    \end{itemize}
\end{frame}


\begin{frame}{Konjunktiv {\scriptsize \cite[S.~99]{stenningHumanReasoningCognitive2008}}}
    \emph{Every card \alert{should} have a vowel on one side and an even number on the other.}
    ($p$ \raisebox{-0.25em}{$\stackrel{\to}{\sim}$} $q$)

    \begin{itemize}
        \item Verstärkung deontischer Interpretation
        \item Suche nach \emph{verletzenden} Karten offensichtlich
        \item Achtung: \emph{Demand Characteristic} \\
            $\to$ Verlangen Karte umzudrehen \\
            $\Rightarrow$ $p,~q$ wählen
        \item deontische Interpretation trotz syntaktisch indikativer Regel
    \end{itemize}
\end{frame}
