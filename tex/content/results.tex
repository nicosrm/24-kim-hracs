% !TeX root = ../main.tex

\section{Resultate}

\begin{frame}{Resultate {\scriptsize \cite[S.~101]{stenningHumanReasoningCognitive2008}}}
    \includegraphics[width=\textwidth]{../plot/results_correct.pdf}
\end{frame}


\begin{frame}{Resultate {\scriptsize \cite[S.~101-102]{stenningHumanReasoningCognitive2008}}}
    \begin{itemize}
        \item jede Manipulation bringt (unterschiedlich gute) Verbesserung
        \item entsprechend Vorhersagen der Theorien, von denen Manipulationen abgeleitet wurden
        \item starke Evidenz für Zusammenhang zwischen Manipulationen und mentalen Vorgängen, auf die sie einwirken
    \end{itemize}
\end{frame}


\begin{frame}{Two-Rule Task {\scriptsize \cite[S.~102-104]{stenningHumanReasoningCognitive2008}}}
    $(p \to q) \leadsto (\{U, I\} \to 8)$

    \begin{itemize}
        \item Verwechslungsgefahr zwischen
        \begin{itemize}
            \item \emph{this rule is \alert{true} of this card}
            \item \emph{this card makes this rule \alert{true}}
        \end{itemize}
        
        \item $q~\hat=~8$ könnte beide Regeln bestätigen \\
            $\Rightarrow$ Konflikt: eine Regel wahr, eine falsch
        
        \item asymmetrische Beziehung von \emph{true}
        
        \item Information-Gain: $\lnot q$ bietet am meisten Informationen \\
            $\Rightarrow$ Vorhersage: häufigste Wahl
    \end{itemize}
\end{frame}


\begin{frame}{Two-Rule Task {\scriptsize \cite[S.~109]{stenningHumanReasoningCognitive2008}}}
    \includegraphics[width=\textwidth]{../plot/results_two_rule.pdf}
\end{frame}


\begin{frame}{Kontingenz {\scriptsize \cite[S.~109]{stenningHumanReasoningCognitive2008}}}
    \includegraphics[width=\textwidth]{../plot/results_contingency.pdf}
\end{frame}


\begin{frame}{Kontingenz {\scriptsize \cite[S.~105,109]{stenningHumanReasoningCognitive2008}}}
    Kontingenz $\Rightarrow$ Entscheidung vor weiteren Informationen
    \begin{itemize}
        \item Beobachtung: prognostizierte Wirkung tritt ein {\small (häufiger $\lnot q$)} \\
            $\Rightarrow$ Erklärung für schwerwiegende Unterschiede
        
        \item problematisch für rationale Analysemodell {\small (u.a. Information-Gain)}
        \begin{itemize}
            \item Warum steigt Wahl von $p,~\lnot q$ signifikant? (15,7\% statt 3,6\%)
            \item Einfluss von Kontingenzen der Antworten auf Interpretation schwierig zu erkennen
        \end{itemize}
    \end{itemize}
\end{frame}


\begin{frame}{Interaktive Kontingenz {\scriptsize \cite[S.~105,109]{stenningHumanReasoningCognitive2008}}}
    \begin{itemize}
        \item \textsc{van Denderen}: reaktive Planung als Quelle der Schwierigkeiten
        \item Subjekte dürfen Karten vor Entscheidung drehen (GUI)
        \item Trick: erste Karte falsifiziert nie
        
        \pause
        \item Kompetenz-Antwort $(p,~\lnot q)$ nun häufigste Antwort \\
        (26\% statt 3,6\%) \\
        $\Rightarrow$ Verlangen nach reaktiver Planung deutlich problematisch
        
        \item[!] mehrere Veränderungen der Aufgabe
        \item[$\Rightarrow$] Verifikation / Falsifikation
        \item[$\Rightarrow$] Unterschied zwischen deskriptiver und deontischer Aufgabe:\\
            (keine) Kontingenzen zwischen Antworten
    \end{itemize}
\end{frame}


\begin{frame}{Truthfulness {\scriptsize \cite[S.~109]{stenningHumanReasoningCognitive2008}}}
    \includegraphics[width=\textwidth]{../plot/results_truthfulness.pdf}
\end{frame}


\begin{frame}{Truthfulness {\scriptsize \cite[S.~107]{stenningHumanReasoningCognitive2008}}}
    \begin{itemize}
        \item Quelle der Regel $\ne$ Experimentator:in
        \item Proband:innen sollen nach Gegenbeispielen suchen
        \item (kleine) signifikante Verbesserung (12,5\% statt 3,6\%)
        \item Ursachen nicht ganz klar
        % TODO: ggf. weiter aufbereiten
    \end{itemize}
\end{frame}


\begin{frame}{Konjunktion {\scriptsize \cite[S.~109]{stenningHumanReasoningCognitive2008}}}
    \includegraphics[width=\textwidth]{../plot/results_conjunction.pdf}
\end{frame}


\begin{frame}{Konjunktion {\scriptsize \cite[S.~107-109]{stenningHumanReasoningCognitive2008}}}
    \begin{itemize}
        \item Formbarkeit der Semantik von Sätzen
        \item Beobachtung: Interpretation ähnlich zu \texttt{if ... else ...}
        \item signifikante Steigerung der Kompetenzantwort {\small (13\% statt 3,6\%)}
        
        \pause
        \item mögliche Ursachen:
        \begin{itemize}
            \item vermehrt deontische Interpretation $\Rightarrow$ Suche nach Verstößen
            \item \textsc{Fillenbaum} zeigt: ca. \sfrac{1}{2} lesen \texttt{if else} {\small (nicht ausreichend)}
            \item Annahme $p$ wahr, entsprechende Folgerung
        \end{itemize}
        
        \pause
        \item Probleme
        \begin{itemize}
            \item deontisches Lesen indikativer Regeln
            \item Unnatürlichkeit, Satz statt Experimentator:in zu hinterfragen
            \item ähnlich schwere (verschiedene, verwandte) Probleme wie Implikation
        \end{itemize}
    \end{itemize}
\end{frame}


\begin{frame}{Konjunktivisch {\scriptsize \cite[S.~109]{stenningHumanReasoningCognitive2008}}}
    \includegraphics[width=\textwidth]{../plot/results_subjunctive.pdf}
\end{frame}


\begin{frame}{Konjunktivisch {\scriptsize \cite[S.~110]{stenningHumanReasoningCognitive2008}}}
    \begin{itemize}
        \item \emph{Every card \alert{should} have a vowel...}
        \item nicht ausreichend, um deontische Interpretation hervorzurufen
        \item alternative epistemische\footnote[frame]{\scriptsize erkenntnistheoretisch} Interpretationen von \emph{should}
        \item deontische Interpretation benötigt inhaltliche Unterstützung
        
        \item Fehlschlag der einfachsten Manipulation zum Test der These
        \begin{itemize}
            \item Interpretation als Beweis gegen Theorie?
            \item \textsc{Stenning} und \textsc{Lambalgen}: weitere Untersuchungen notwendig
        \end{itemize}
    \end{itemize}
\end{frame}


\begin{frame}{Putting it all Together {\scriptsize \cite[S.~110-112]{stenningHumanReasoningCognitive2008}}}
    \begin{itemize}
        \item Ergebnisse entsprechen Tutoren-Dialoge (Kapitel 3)
        \item Manipulationen senken Interpretationsschwierigkeiten \\
            $\Rightarrow$ Indiz für Beitrag verschiedener Problemquellen
        \item semantische Herausforderungen deskriptiver Aufgaben
        
        \item entgegen Theorien\footnote[frame]{\scriptsize außer Information-Gain (aber andere Probleme)}, die annehmen, dass
        \begin{itemize}
            \item klassische Logik die korrekte Interpretation ist und
            \item dies Interpretation der Proband:innen sei
        \end{itemize}
    \end{itemize}
\end{frame}
