% !TeX root = ../main.tex

{
    \metroset{sectionpage=none}
    \section{Einführung}
}

\begin{frame}{Einführung {\scriptsize \cite[S.~93-95]{stenningHumanReasoningCognitive2008}}}
    \begin{itemize}
        \item Interpretationsschwierigkeiten in Sokratischen Dialogen \\
            $\not \Rightarrow$ Auftreten in originaler \textsc{Wason}-Aufgabe

        \item These 1: Interpretation $\lightning$ vorgelegte Materialien \\
            $\Rightarrow$ aufgezeigtes Verhalten $\centernot {\hat=}$ klassisches Logikmodell

        \item These 2: wichtiger Faktor: deskriptive oder deontische Interpretation

        \item Basis: Standard-\textsc{Wason} mit \alert{Manipulationen} \\
            $\Rightarrow$ Untersuchung des Beitrags verschiedener Faktoren
        
        \item Ziel: Minderung der Konflikte ($\hookrightarrow$ Schwierigkeiten), \\
            Hervorrufen von erwarteter Reaktion von \textsc{Wason}
    \end{itemize}
\end{frame}


\begin{frame}{Wason's Selection Task {\scriptsize \cite[S.~44-46]{stenningHumanReasoningCognitive2008}}}
    \textbf{Regel:} \emph{If there is a vowel on one side, then there is an even number on the other side.}

    \[
        \framebox[5em][c]{A\strut}\quad
        \framebox[5em][c]{K\strut}\quad
        \framebox[5em][c]{4\strut}\quad
        \framebox[5em][c]{7\strut}
    \]

    Welche Karten \emph{müssen} umgedreht werden, um zu entscheiden, ob die Regel wahr ist?

    $\Rightarrow (p \to q) \leadsto (\text{Vokal} \to 2\mathbb{N})$

    Kompetenzantwort: $p$ und $\lnot q$, d.h. A und 7
\end{frame}

\begin{frame}{Wason's Selection Task {\scriptsize \cite[S.~44-46]{stenningHumanReasoningCognitive2008}}}
    $$(p \to q) \leadsto (\text{Vokal} \to 2\mathbb{N})$$

    \begin{center}
        \begin{tabular}{c | c | c | c | c}
            $p$ &$p,~q$ &\alert{$p, ~\lnot q$} &$p, ~q, ~\lnot q$ &anderes \\
            \hline
            A &A, 4 &\alert{A, 7} &A, 4, 7 & \\
            \hline \hline
            35\% &45\% &\alert{5\%} &7\% &8\%
        \end{tabular}
    \end{center}
\end{frame}


\begin{frame}{Two-Rule Task {\scriptsize \cite[S.~61-63]{stenningHumanReasoningCognitive2008}}}
    \begin{enumerate}
        \item \emph{If there is a $U$ on one side, then there is an 8 on the other side.}
        \item \emph{If there is a $I$ on one side, then there is an 8 on the other side.}
    \end{enumerate}

    \[
        \framebox[5em][c]{U\strut}\quad
        \framebox[5em][c]{I\strut}\quad
        \framebox[5em][c]{8\strut}\quad
        \framebox[5em][c]{3\strut}
    \]

    Genau eine Regel ist wahr. Welche Karten \emph{müssen} umgedreht werden, um zu entscheiden, welche Regel wahr ist?

    $\Rightarrow (p \to q) \leadsto (\{U, I\} \to 8)$

    Kompetenzantwort: $\lnot q$, d.h. $3$\\
    {\footnotesize (hinreichend, identifiziert falsche Regel)}
\end{frame}
