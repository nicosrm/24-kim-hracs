% !TeX root = ../main.tex

{
    \metroset{sectionpage=none}
    \section{Einführung}
}

\begin{frame}{Einführung}
    Lorem ipsum
\end{frame}

\begin{frame}{Wason's Selection Task {\scriptsize \cite[S.~44-46]{stenningHumanReasoningCognitive2008}}}

    \textbf{Regel:} \emph{If there is a vowel on one side, then there is an even number on the other side.}

    \[
        \framebox[5em][c]{A\strut}\quad
        \framebox[5em][c]{K\strut}\quad
        \framebox[5em][c]{4\strut}\quad
        \framebox[5em][c]{7\strut}
    \]

    Welche Karten \emph{müssen} umgedreht werden, um zu entscheiden, ob die Regel wahr ist?

    $\Rightarrow (p \to q) \rightsquigarrow (\text{Vokal} \to 2\mathbb{N})$

    Kompetenzantwort: $p$ und $\lnot q$, d.h. A und 7
\end{frame}

\begin{frame}{Wason's Selection Task {\scriptsize \cite[S.~44-46]{stenningHumanReasoningCognitive2008}}}
    $(p \to q) \rightsquigarrow (\text{Vokal} \to 2\mathbb{N})$

    \begin{center}
        \begin{tabular}{c | c | c | c | c}
            $p$ &$p,~q$ &\alert{$p, ~\lnot q$} &$p, ~q, ~\lnot q$ &anderes \\
            \hline
            A &A, 4 &\alert{A, 7} &A, 4, 7 & \\
            \hline \hline
            35\% &45\% &\alert{5\%} &7\% &8\%
        \end{tabular}
    \end{center}
\end{frame}
