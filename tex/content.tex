% !TeX root = main.tex

\begin{frame}{Zuvor bei HRaCS...}
    \begin{itemize}
        % Chapter 1
        \item Logik \& Psychologie
        \item normative, deskriptive, präskriptive Regeln
        \pause
        
        % Chapter 2
        \item Konzepte der modernen Logik, Klassische Aussagenlogik, Closed-World-Reasoning etc.
        \item großer Spielraum für Interpretationen von Konditionalen
        \pause

        % Chapter 3
        \item Wason's Selection Task, Two-Rule-Task, Tutoren-Dialoge
        \item Bayes'sches Information-Gain Model
        \item[$\Rightarrow$] \alert{Interpretationsschwierigkeiten} von Proband:innen
        \begin{itemize}
            \item v.a. bei (abstrakten) deskriptiven Konditionalen
            \item kaum/nicht bei deontischen Regeln mit vertrauten Thematiken
        \end{itemize}
    \end{itemize}
\end{frame}

\begin{frame}{Gliederung}
    \tableofcontents
  \end{frame}

\section{First Section}

\begin{frame}{First Frame}
    Hello, world! \cite{stenningHumanReasoningCognitive2008}
\end{frame}
